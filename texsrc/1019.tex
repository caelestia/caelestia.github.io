\documentclass{article}

\usepackage{geometry}
\usepackage{amsfonts, amsmath, amssymb, amsthm}
\usepackage{enumitem}
\usepackage{commands}
\usepackage{mathtools}
\usepackage{hyperref}
\usepackage{tikz-cd}
\hypersetup{colorlinks, linkcolor={blue}, citecolor={blue}, urlcolor={blue}}


\begin{document}
\puttitle{Nakayama's Lemma}\self

This note tries to cover some basic commutative algebra. So \emph{all rings are commutative}.

Indeed, Nakayama's lemma is used to reduce finite modules to linear algebra. Let's start with the linear algebra.

\begin{theorem}[Hamilton-Cayley]
    Let $R$ be a ring and let $A\in\Mat{n}(R)$. Then $\chi_A(A)=0$.
\end{theorem}

\begin{proof}
    There are of course a ton of ways to prove this. The most standard proof is
    \[
        \chi_A(t):=\det(t-A) \ \Longrightarrow\ \chi_A(A)=\det(0)=0.
    \]
    In more details, let $Y=R[t]$ be a formal polynomial ring, and $V:=R^{\oplus n}$ has a $Y$-module structure given by $t\leadsto A$. Consider the base change $V_Y=Y^{\oplus n}$ of $V$ and $t-A\in\End(V_Y)$. Observe that the composition
    \[ \begin{tikzcd}
        V \ar[r, hookrightarrow] & V_Y \ar[r] & \coker{(t-A)}
    \end{tikzcd} \]
    is an isomorphism of $Y$-modules. Thus $\chi_A(t)=(t-A)^{\vee}(t-A)$ annihilates $V$.
\end{proof}

\begin{remark}
    Alternatively, observe that this is a purely formal proposition and it suffices to give a proof over the field $\Q(X_{11},X_{12},\cdots,X_{nn})$. Since a matrix $A$ is non-diagonalizable if and only if the discriminant of the determinant of $t-A$ vanishes, the set these matrices is a Zariski closed set. Indeed, an infinite field, every nonempty Zariski open set is dense. Hence we may reduce to the case where $A$ is diagonal, which is trivial.
\end{remark}

\section{Nakayama}

\begin{theorem}[Nakayama's lemma, 1st version]
    Let $R$ be a ring and $I$ be an ideal. If $M$ is a finitely generated module satisfying $IM=M$, then there exists $\alpha\in 1+I$ which annihilates $M$.
\end{theorem}

\begin{proof}
    Pick a set $\bc{e_i}_{i=1}^n$ of generators of $M$. The condition implies elements $a_{ij}\in I$ satisfying
    \[
        e_i = \sum_{j} a_{ij}e_j,\quad \forall i.
    \]
    
\end{proof}

\section*{The End}



\noindent Compiled on \todayymd.

\noindent\home

\end{document}
