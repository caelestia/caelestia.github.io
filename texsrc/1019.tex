\documentclass{article}

\usepackage{geometry}
\usepackage{amsfonts, amsmath, amssymb, amsthm}
\usepackage{enumitem}
\usepackage{commands}
\usepackage{mathtools}
\usepackage{hyperref}
\hypersetup{colorlinks, linkcolor={blue}, citecolor={blue}, urlcolor={blue}}


\begin{document}
\puttitle{Nakayama's Lemma}\self

This note tries to cover some basic commutative algebra. So \emph{all rings are commutative}.

Indeed, Nakayama's lemma is used to reduce finite modules to linear algebra. Let's start with the linear algebra.

\begin{theorem}[Hamilton-Cayley]
    Let $R$ be a ring and let $A\in\Mat{n}(R)$. Then $\chi_A(A)=0$.
\end{theorem}

\begin{proof}
    There are of course a ton of ways to prove this. The most standard proof is
    \[
        \chi_A(t):=\det(t-A) \ \Longrightarrow\ \chi_A(A)=\det(0)=0.
    \]
    In more details, let $Y=R[t]$ be a formal polynomial ring, and $V:=R^{\oplus n}$ has a $Y$-module structure given by $t\leadsto A$. Consider the base change $V_Y=Y^{\oplus n}$ of $V$ and $t-A\in\End(V_Y)$. 
\end{proof}

\section{Nakayama}

\section*{The End}



\noindent Compiled on \todayymd.

\noindent\home

\end{document}
