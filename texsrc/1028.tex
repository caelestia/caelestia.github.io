\documentclass{article}

\usepackage{geometry}
\usepackage{amsfonts, amsmath, amssymb, amsthm}
\usepackage{enumitem}
\usepackage{commands}
\usepackage{mathtools}
\usepackage{hyperref}
\hypersetup{colorlinks, linkcolor={blue}, citecolor={blue}, urlcolor={blue}}


\begin{document}
\puttitle{Most continuous functions are nowhere differentiable}\self

The topic of this note is simply to show the Baire category theorem (BCT), which we state now.
\begin{theorem}[Baire, Hausdorff]
    Let $(X,d)$ be a complete metric space, $\bc{U_n}_{n\geq 1}$ be a sequence of dense open subsets of $X$. Then the intersection of all the $U_n$ is dense.
\end{theorem}

\begin{proof}
    Suppose that $V\neq\varnothing$ is an open set disjoint from $\bigcap U_n$. We inductively define two sequences $\bc{x_i}$ and $\bc{r_i}$, satisfying
    \begin{itemize}[itemsep=0.2ex]
        \item $B(x_1,r_1)\subset V$,
        \item $B(x_{i+1},r_{i+1})\subset B(x_{i},\frac{1}{2}r_i)\cap U_i$,
        \item $r_i\leq 2^{-i}$ for all $i\geq 1$.
    \end{itemize}
    This makes $B(x_i,r_i)$ a strictly descending sequence of open balls.

    To conclude the proof, we need to show that $x_i$ is a Cauchy sequence, that its limit $x$ lies in $U_n$, and that its limit also lies in $V$.
    This is clear, since we have
    \[
        x_k \in B(x_i,r_i), \quad \forall k\geq i\geq 1,
    \]
    \[
        \Longrightarrow\ x \in \overline{B(x_i,r_i)} \subset \overline{B(x_{i-1},\tfrac{1}{2}r_{i-1})} \subset D(x_{i-1},\tfrac{1}{2}r_{i-1}) \subset B(x_{i-1},r_{i-1})
    \]
    for $i\geq 2$. Now, the last ball in the chain lies in $V$ if $i=2$, and in $U_{i-2}$ otherwise.
\end{proof}

\begin{remark}
    The result holds under a different assumption that $X$ is locally compact Hausdorff.
\end{remark}

Next, we give an application 

\section*{The End}



\noindent Compiled on \todayymd.

\noindent\home

\end{document}
