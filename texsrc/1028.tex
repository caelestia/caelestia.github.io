\documentclass{article}

\usepackage{geometry}
\usepackage{amsfonts, amsmath, amssymb, amsthm, mathrsfs}
\usepackage{enumitem}
\usepackage{commands}
\usepackage{mathtools}
\usepackage{hyperref}
\hypersetup{colorlinks, linkcolor={blue}, citecolor={blue}, urlcolor={blue}}

\newcommand{\conts}{\cat{C}^0([0,1],\R)}

\begin{document}
\puttitle{Most continuous functions are nowhere differentiable}\self

The topic of this note is simply to show the Baire category theorem (BCT), which we state now.
\begin{theorem}[Baire, Hausdorff]
    Let $(X,d)$ be a complete metric space, $\bc{U_n}_{n\geq 1}$ be a sequence of dense open subsets of $X$. Then the intersection of all the $U_n$ is dense.
\end{theorem}

\begin{proof}
    Suppose that $V\neq\varnothing$ is an open set disjoint from $\bigcap U_n$. We inductively define two sequences $\bc{x_i}$ and $\bc{r_i}$, satisfying
    \begin{itemize}[itemsep=0.2ex]
        \item $B(x_1,r_1)\subset V$,
        \item $B(x_{i+1},r_{i+1})\subset B(x_{i},\frac{1}{2}r_i)\cap U_i$,
        \item $r_i\leq 2^{-i}$ for all $i\geq 1$.
    \end{itemize}
    This makes $B(x_i,r_i)$ a strictly descending sequence of open balls.

    To conclude the proof, we need to show that $x_i$ is a Cauchy sequence (obvious), that its limit $x$ lies in all $U_n$, and that the limit $x$ lies in $V$.
    We have
    \[
        x_k \in B(x_i,r_i), \quad \forall k\geq i\geq 1,
    \]
    \[
        \Longrightarrow\ x \in \overline{B(x_i,r_i)} \subset \overline{B(x_{i-1},\tfrac{1}{2}r_{i-1})} \subset D(x_{i-1},\tfrac{1}{2}r_{i-1}) \subset B(x_{i-1},r_{i-1})
    \]
    for $i\geq 2$. Now, the last term in the chain lies in $V$ if $i=2$, and in $U_{i-2}$ otherwise.
\end{proof}

\begin{remark}
    The result holds under a different assumption that $X$ is locally compact Hausdorff.
\end{remark}

\section{Main result}

Next, we give an application which is the claim in the title: ``most'' continuous functions are differentiable.

First, we need to find an appropriate notion of being a ``big'' subset of the space of continuous functions. This space is too \emph{big} (for example, it's not locally compact) to have a good measure, like the Lebesgue measure on $\R^n$. I believe this ``issue'' is handled to some extent in functional analysis, and namely, by the BCT. What we will prove is the following.
\begin{theorem}
    Let $\cal{A}$ be the subset of $\conts$ consisting of the functions that are nowhere differentiable on $(0,1)$. Under the uniform norm, $\cal{A}$ is dense in $\conts$.
\end{theorem}

I'll prove this in two steps.

\subsection*{Step 1. more continuous than continous functions}

As will be clear from the second step of the proof, we want these functions to be at least Lipchitz continuous, for us to do basically anything. 

Thanks to the following theorem, we can dodge the danger of losing every brain cell from considering general continuous functions.

\begin{theorem}[Stone-Weierstrass]
    Polynomial functions are dense in $\conts$.
\end{theorem}

\begin{proof}
    Given in Section \ref{pf}.
\end{proof}

\subsection*{Step 2. Apply BCT}

A function in $\conts$ is differentiable at at least one point in (0,1) only if it lies in the union of
\[
    X_{n,m} := \bc{f:\exists x\in(0,1)\text{ s.t. }0<\abs{y-x}<\tfrac{1}{m}\implies\frac{\abs{f(y)-f(x)}}{\abs{y-x}}<n},
\]
so $A$ is the intersection of all $Y_{n,m}:=(X_{n,m})^{\mathrm{c}}$. By the BCT, it suffices to show that each $Y_{n,m}$ is dense. 



\section{Proof of Stone-Weierstrass}\label{pf}

\section*{The End}



\noindent Compiled on \todayymd.

\noindent\home

\end{document}
