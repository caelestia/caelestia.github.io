\documentclass{article}

\usepackage{geometry}
\usepackage{amsfonts, amsmath, amssymb, amsthm}
\usepackage{enumitem}
\usepackage{commands}
\usepackage{mathtools}
\usepackage{hyperref}
\usepackage{tikz-cd}
\hypersetup{colorlinks, linkcolor={blue}, citecolor={blue}, urlcolor={blue}}

\newcommand{\rad}{\operatorname{rad}}

\begin{document}
\puttitle{Nakayama's Lemma}\self

This note tries to cover some basic commutative algebra. So \emph{all rings are commutative}.

Indeed, Nakayama's lemma is used to reduce finite modules to linear algebra. Let's start with the linear algebra.

\begin{theorem}[Hamilton-Cayley]
    Let $R$ be a ring and let $A\in\Mat{n}(R)$. Then $\chi_A(A)=0$.
\end{theorem}

\begin{proof}
    There are of course a ton of ways to prove this. The most standard proof is
    \[
        \chi_A(t):=\det(t-A) \ \Longrightarrow\ \chi_A(A)=\det(0)=0.
    \]
    In more details, let $Y=R[t]$ be a formal polynomial ring, and $V:=R^{\oplus n}$ has a $Y$-module structure given by $t\leadsto A$. Consider the base change $V_Y=Y^{\oplus n}$ of $V$ and $t-A\in\End(V_Y)$. Observe that the composition
    \[\begin{tikzcd}
        V \ar[r, hookrightarrow] & V_Y \ar[r] & \coker{(t-A)}
    \end{tikzcd}\]
    is an isomorphism of $Y$-modules. Thus $\chi_A(t)=(t-A)^{\vee}(t-A)$ annihilates $V$.
\end{proof}

\begin{remark}
    Alternatively, observe that this is a purely formal proposition and it suffices to give a proof over the field $\Q(X_{11},X_{12},\cdots,X_{nn})$. Since a matrix $A$ is non-diagonalizable if and only if the discriminant of the determinant of $t-A$ vanishes, the set these matrices is a Zariski closed set. Indeed, an infinite field, every nonempty Zariski open set is dense. Hence we may reduce to the case where $A$ is diagonal, which is trivial.
\end{remark}

\section{Nakayama}

\begin{theorem}[Nakayama's lemma, 1st version]\label{thm:naka}
    Let $R$ be a ring and $I$ be an ideal. If $M$ is a finitely generated $R$-module, we have
    \begin{enumerate}[itemsep=1pt, label=(\alph*)]
        \item If $\varphi\in\End(M)$ and $\varphi(M)\subset IM$, then $\varphi$ satisfies a polynomial equation $X^n+\sum_{d=0}^{n-1}a_dX^d=0$, where $a_d\in I^{n-d}$.
        \item If $IM=M$, then there exists $\alpha\in 1+I$ which annihilates $M$.
    \end{enumerate}
\end{theorem}

\begin{proof}
    (b) follows from (a) by taking $\varphi=\id_M$. Pick a set $\bc{e_i}_{i=1}^n$ of generators of $M$. Using condition of (a) we have elements $a_{ij}\in I$ satisfying
    \[
        \varphi(e_i) = \sum_{j} a_{ij}e_j,\quad \forall i.
    \]
    In other words, the matrix $A=\bp{a_{ij}}$ makes the following diagram commute:
    \[\begin{tikzcd}
        R^{\oplus n} \ar[rr, "A"] \ar[rd, two heads] & & M^{\oplus n} \ar[ld, two heads] \\
        & M &
    \end{tikzcd}\]
    where the two surjections are both given by $(x_i)_i \mapsto \sum x_ie_i\in M$. The polynomial $\chi_A(t)$ has the required form, and by Hamilton-Cayley, $\chi_A(\varphi)=0\in\End_R(M)$.
\end{proof}

\begin{proposition}
    Let $M$ be a finitely generated $R$-module. If $\varphi:M\to M$ is surjective, then it is an isomorphism.
\end{proposition}

\begin{proof}
    $M$ has a finitely generated $R[t]$-module structure given by $t\leadsto\varphi$. In Theorem \ref{thm:naka} (b), take $I=(t)$. The condition $IM=M$ follows from surjectivity of $\varphi$. We conclude that $1+tf(t)$ annihilates $M$ for some $f(t)\in R[t]$, and the result follows.
\end{proof}

\begin{definition}
    The Jacobson radical in a ring $R$ is the intersection of all maximal ideals, denoted by $\rad(R)$. 
\end{definition}

\begin{lemma}
    If $I$ is an ideal of $R$, then $I\subset\rad(R)$ if and only if every element of $1+I$ is a unit.
\end{lemma}

\begin{proof}
    If $\beta\in\rad(R)$, then $1+\beta$ does not lie in any maximal ideal, so it must be a unit. Conversely, if $I\not\subset\mathfrak{m}$ for some maximal ideal $\mathfrak{m}\subset R$, then the quotient map $R\to R/\mathfrak{m}$ is surjective on $I$. In particular, there exists $\beta\in I$ such that $1+\beta\in\mathfrak{m}$.
\end{proof}

\begin{corollary}
    Let $M$ be a finitely generated $R$-module. If $I\subset\rad(R)$ and $IM=M$, then $M=0$.
\end{corollary}

\begin{proof}
    By Theorem \ref{thm:naka} (b).
\end{proof}

Put $k:=R/I$ and denote the base change as $(-)_k:=(-)\otimes k$, even when $k$ if not a field.

\begin{corollary}[Nakayama's lemma, 2nd version]
    Let $M$, $N$ be $R$-modules, $I\subset\rad(R)$. Suppose that $M$ is finitely generated. We have
    \begin{enumerate}[itemsep=1pt, label=(\alph*)]
        \item If $M_k=0$, then $M=0$.
        \item If $\varphi\in\hom_R(N,M)$ and $\varphi_k$ is surjective, then so is $\varphi$.
        \item A subset $\bc{x_i}_{i=1}^n\subset M$ generates $M$ if and only if $\bc{\bar x_i}_i$ generates $M_k$.
    \end{enumerate}
\end{corollary}

\begin{proof}
    If $M_k=M\otimes(R/I)=M/IM=0$, then $M=0$ by the last corollary. This proves (a). Now tensor the following SES with $k$,
    \[\begin{tikzcd}
        0 \ar[r] & N \ar[r] & M \ar[r] & \coker \ar[r] & 0
    \end{tikzcd}\]
    And we see that $\coker{}\!\otimes k=0$. Clause (b) then follows from (a). Finally, we show (c) by applying (b) to the map $R^{\oplus n}\longrightarrow M$.
\end{proof}

\section{An application}

Let $(R,\mathfrak{m},k)$ be a Noetherian local ring, and $M$ be a finitely generated flat $R$-module. Then $M$ is in fact free.

To see this, lift a basis of $M_k$ to a minimal set of generators of $M$ using (c), and consider the sequence
\[\begin{tikzcd}
    0 \ar[r] & \ker \ar[r] & R^{\oplus n} \ar[r] & M \ar[r] & 0 
\end{tikzcd}\]
On tensoring with $k$, since $\mathrm{Tor}_1^R(M,k)=0$, we obtain an exact sequence of $k$-vector spaces
\[\begin{tikzcd}
    0 \ar[r] & \ker{}\!\otimes k \ar[r] & k^{\oplus n} \ar[r,"\sim"] & M_k \ar[r] & 0 
\end{tikzcd}\]
So $\ker{}\!\otimes k=0$. $\ker{}\!$ is finitely generated as a submodule of $R^{\oplus n}$, thus (a) implies that $\ker{}\!=0$, as desired.


\section*{The End}



\noindent Compiled on \todayymd.

\noindent\home

\end{document}
