\documentclass{article}

\usepackage{geometry}
\usepackage{amsfonts, amsmath, amssymb, amsthm}
\usepackage{enumitem}
\usepackage{commands}
\usepackage{mathtools}
\usepackage{hyperref}
\hypersetup{colorlinks, linkcolor={blue}, citecolor={blue}, urlcolor={blue}}


\begin{document}
\puttitle{Fundamental theorem of symmetric polynomials}\self

Let $k$ be a field, and let $K=k(X_1,X_2,\cdots,X_n)$ be the field of rational functions in $n$ variables. $S_n$ naturally acts on $K$ by permuting the indeterminates. We then view $S_n$ as a subset of $\Aut(K)$.

\begin{lemma}
    The extension $K\vert K^{S_n}$ is Galois.
\end{lemma}

\begin{proof}
    This extension is the splitting field of the polynomial
    \begin{equation}\label{eq:f}
        f(X) = (X-X_1)(X-X_2)\cdots(X-X_n) \in K^{S_n}[X],
    \end{equation}
    which clearly has no repeated roots.
\end{proof}

We could easily check that $S_n=\Gal(K\vert K^{S_n})$. But for an obvious generalization of this, we actually need to take a slightly tricky path.

\begin{theorem}\label{thm.K^G}
    Let $K$ be a field and let $G\leq\Aut(K)$ be a finite subgroup. Then $K\vert K^G$ is Galois, and $\Gal(K\vert K^G)=G$.
\end{theorem}

\begin{proof}
    Mimicking \eqref{eq:f}, for every $x\in K$, we wish to write
    \begin{equation*}
        g_x(X) \overset{?}{=} \prod_{h\in G}(X-h(x)) \in (K[X])^G=K^G[X].
    \end{equation*}
    But this doesn't work, because $g_x(X)$ might has repeated roots. The correct solution is to use the $G$-orbit of $x$:
    \begin{equation*}
        g_x(X) = \prod_{y\in Gx}(X-y) \in (K[X])^G=K^G[X].
    \end{equation*}
    This time, we see that any simple subextension $K^G(x)\vert K^G$ is the splitting field of $g_x(X)$, and thus is Galois. We have an upper bound $[K^G(x):K^G] \leq \vert G\vert$. Clearly, every finite subextension of $K\vert K^G$ is separable, so the primitive element theorem always applies. Combining with the above, we see that every finite subextension has degree $\leq \vert G\vert$. This means that $K\vert K^G$ is itself finite (hence Galois), and $[K:K^G] = \vert\Gal(K\vert K^G)\vert \leq \vert G\vert$. As $G\subset\Gal(K\vert K^G)$, the result follows.
\end{proof}

The coeffients of $f$ are called the \emph{elementary symmetric polynomials} in $X_1,X_2,\cdots,X_n$, and are denoted by $\sigma_1,\sigma_2,\cdots,\sigma_n$, the subscripts indicating their degree. We should be aware of what this means:

\begin{proposition}
    $K^{S_n}$ is generated by the elementary symmetric polynomials $\sigma_1,\sigma_2,\cdots,\sigma_n$.
\end{proposition}

\begin{proof}
    Let $L=k(\sigma_1,\sigma_2,\cdots,\sigma_n)\subset K^{S_n}$. Then $K$ is the splitting field of $f(X)$ over $L$ as well. We obtain a bound
    \begin{equation*}
        [K:K^{S_n}] \leq [K:L] \leq n!.
    \end{equation*}
    By Theorem \ref{thm.K^G}, $[K:K^{S_n}]=\vert S_n\vert=n!$. Therefore $L=K^{S_n}$.
\end{proof}

Recall that an integral domain $R$ is \emph{integrally closed} if any element of the field $\Frac(R)$ which is a root of a monic polynomial in $R[X]$ must lie in $R$.

\begin{lemma}
    Let $R=k[X_1,X_2,\cdots,X_n]$ be the polynomial ring. Then $R$ is integrally closed.
\end{lemma}

\begin{proof}
    Brings back the bittersweet middle school memories!
\end{proof}

\begin{remark}
    For the same reason, any GCD domain is integrally closed. The lemma is a special case of this.
\end{remark}

\begin{theorem}[Fundamental theorem of symmetric polynomials]
    We have an equation of subalgebras of $L$ and $K^{S_n}$:
    \begin{equation*}
        k[\sigma_1,\sigma_2,\cdots,\sigma_n] = K^{S_n}\cap k[X_1,X_2,\cdots,X_n].
    \end{equation*}
\end{theorem}

\begin{proof}
    Since $n=\trdeg(K\vert k)=\trdeg(K^{S_n}\vert k)$, the $n$ generators $\sigma_1,\sigma_2,\cdots,\sigma_n$ of $K^{S_n}$ must be algebraically independent. Therefore $R:=k[\sigma_1,\sigma_2,\cdots,\sigma_n]$ is a polynomial ring, with $K^{S_n}$ as its field of fraction. By \eqref{eq:f}, the $X_i$'s are integral over $R$. It follows that $k[X_1,X_2,\cdots,X_n]$ is integral over $R$ for it's a commutative ring. This completes the proof.
\end{proof}

\section*{The End}

Final remark: I don't know any commutative algebra. Very inconvenient.

\hfill

\noindent Compiled on \todayymd.

\noindent\home

\end{document}
