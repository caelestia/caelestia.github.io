\documentclass{article}

\usepackage{geometry}
\usepackage{amsfonts, amsmath, amssymb, amsthm}
\usepackage{enumitem}
\usepackage{commands}
\usepackage{mathtools}
\usepackage{hyperref}
\usepackage{mathrsfs}
\hypersetup{colorlinks, linkcolor={blue}, citecolor={blue}, urlcolor={blue}}


\begin{document}
\puttitle{Basic example \& existence of non-measureable sets}\self

We will consider the Borel $\sigma$-algebra on $\mathbb{R}$ and its completion. The measure is always denoted by $m$. We begin by reviewing the easiest example of a non-measurable set, the Vitali set.

The idea is easy to follow: we find a countable partition of $[0,1]$ into closely related sets. Choose a subset $A\subset[0,1]$ which contains exactly one element in each class of $\R/\Q$. Let $0\in A$ for convenience. For rational numbers $q\in [0,1)$, define
\[
    A_q = [0,1]\cap\bp{(q+A)\cup(q+A-1)}.
\]
Since $A\subset[0,1)$, $q+A$ and $q+A-1$ are always disjoint. Given that $m$ satisfies translation invariance, we have $m(A_q)=m(A)$ for all $q$.

$A$ is known as the Vitali set. We showed that

\begin{proposition}
    The Vitali set is not Lebesgue (Borel) measurable.
\end{proposition}

\begin{proof}
    The $\bc{A_q}$ form a disjoint partion of the interval $[0,1)$, which has measure $1$.
\end{proof}

Our next goal is to answer the following question---at least partially. Given a subset $E\subset\R$, is there a non-measurable subset of $E$? 

When $E$ is measurable has positive measure, it is actually not more difficult.

\begin{proposition}
    If $E$ is Lebesgue measureable subset of $[0,1]$ and $m(E)>0$, then $E\cap A$ is a non-measurable set.
\end{proposition}

\begin{proof}
    The translations $E_q:=q+E\cap A$, $q\in\Q\cap[0,1)$, are disjoint and their union is a set between $E$ and $[0,2]$. Hence we get the same contradiction as above.
\end{proof}

\section{An existence result}

If $\mathcal{P}(E)$ is too big in size, some subset of $E$ can't be Borel measurable.

\begin{proposition}
    Let $S$ be an infinite subset of $\mathcal{P}(X)$, then the cardinality of the $\sigma$-algebra generated by $S$ satisfies
    \[
        \abs{\mathscr{M}(S)} = \abs{S}\cdot\aleph_1.
    \]
\end{proposition}

\begin{proof}
    For all countable ordinals $\alpha$, inductively define
    \begin{itemize}[nosep]
        \item $M_0$ is the union of $S$ with the set of subsets whose complement is in $S$;
        \item $U_{\alpha}$ denotes the set of countable unions of elements of $M_\alpha$, and we put $M_{\alpha+1}:=U_{\alpha}\cup\bc{B^c:B\in U_{\alpha}}$;
        \item If $\alpha>0$ is a limit ordinal, put $M_\alpha=\bigcup_{\beta<\alpha}M_\beta$.
    \end{itemize}
    Every $M_\alpha$ is closed under taking complements, and has the same cardinality as $\abs{S}$ if $\alpha$ is countable. By a cofinality argument ($\operatorname{cof}(\omega)>\omega$), we see that $M_{\aleph_1}$ is an $\sigma$-algebra, hence it is $\mathscr{M}(S)$. 
\end{proof}

\begin{corollary}
    The Borel $\sigma$-algebra $\mathcal{B}_\R$ has cardinality $\mathfrak{c}:=\abs{\R}$.
\end{corollary}

\begin{proof}
    Yes.
\end{proof}

A classical result from descriptive set theory states that any Borel set of a Polish space has the perfect set property; in particular, an uncountable Borel set in $\R^d$ must have cardinality $\mathfrak{c}$. Therefore it must contain a non-Borel-measurable set.

For other subsets of $\R^d$, as long as its power set has a cardinality larger than $\mathfrak{c}$, this will also hold.

\section*{The End}



\noindent Compiled on \todayymd.

\noindent\home

\end{document}