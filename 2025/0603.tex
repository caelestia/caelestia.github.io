\documentclass{article}

\usepackage{geometry}
\usepackage{amsfonts, amsmath, amssymb, amsthm}
\usepackage{enumitem}
\usepackage{commands}
\usepackage{mathtools}
\usepackage{hyperref}
\hypersetup{colorlinks, linkcolor={blue}, citecolor={blue}, urlcolor={blue}}


\begin{document}
\puttitle{Semisimple modules and rings}\self

\section{Semisimple Modules}

In the first part of this note, we give a criterion for semisimple modules.

\begin{definition}
    An $R$-module $M$, unless stated otherwise, refers to a right $R$-module. Recall that $M$ is simple if its only submodules are $0$ and $M$ itself.
\end{definition} 

Clearly $D:=\End_R(M)$ is a division ring. Also, $\End_R(M^n)\simeq\Mat{n}(D)$.

\begin{definition}\label{def.semisimple}
    We say $M$ is semisimple if it is a direct sum of simple modules, and $R$ is semisimple (as a ring) if all $R$-modules are semisimple. 
\end{definition}

By collecting the isomorphic summands, we have the \emph{isotypic decomposition}
\begin{equation}
    M\simeq \bigoplus_{i\in I}L_i^{d_i},
\end{equation}
where the invariants $\{L_i\}$: mutually non-isomorphic simple submodules, $D_i=\End_R(L_i)$, $d_i>0$.

If $I$ and all $d_i$'s are finite, we see that $d_i=\dim_{D_i}(\hom(L_i,M))$, and
\begin{equation}\label{eq.ssring}
    \End_R (M)\simeq\prod_{i\in I}\Mat{d_i}(D_i)
\end{equation}

\begin{lemma}
    If $M$ is semisimple and $N\subset M$ is a submodule, then $N$ is a direct summand of $M$. 
    In other words, all short exact sequences
    \begin{equation*}
        0 \longrightarrow N \longrightarrow M \longrightarrow M/N \longrightarrow 0
    \end{equation*}
    split.
\end{lemma} 

\begin{proof}
    Let $M=\bigoplus_{i\in I}L_i$ where $L_i$ are simple. 
    With Zorn's lemma, pick a maximal $J\subset I$ such that $N\cap \sum_{j\in J}L_j=\varnothing$.
    To see $N+\sum_{j\in J}L_j=M$, we just need to notice that it contains every $L_i$.
\end{proof}


\begin{corollary}\label{cor.sssubquot}
    Subquotients and sums of semisimple modules are semisimple.
\end{corollary}

\begin{proof}
    First part is clear from the proof. For a sum of semisimple modules $M_i$, note the surjective map $\bigoplus_{i}M_i\twoheadrightarrow\sum_iM_i$.
\end{proof}

\begin{proposition}
    The converse of Lemma 1 is also true.
\end{proposition}

\begin{proof}
    Define the socle of $M$ as 
    \begin{equation*}
        \soc(M)=\sum\left\{N\subset M:\text{simple submodule}\right\}.
    \end{equation*}
    By Corollary \ref{cor.sssubquot}, $\soc(M)$ is the maximal semisimple submodule in $M$. By assumption we have $M\simeq \soc(M)\oplus H$ for some $H\subset M$. The conclusion follows from the following two lemmas.
\end{proof}

\begin{lemma}
    Simple modules.
    \begin{enumerate}[nosep]
        \item A proper submodule $N\subset M$ is maximal if and only if $M/N$ is simple.
        \item Every nonzero finitely generated module has a maximal submodule.
        \item Every nonzero module has a simple subquotient.
    \end{enumerate}
\end{lemma}

\begin{proof}
    1 is trivial. 3 follows from 2 because the cyclic submodule $aR$ is finitely generated.

    For 2, we use Zorn's lemma. Pick $a\in M\setminus\{0\}$. Put
    \begin{equation*}
        \mathcal{S}=\{N\subset M:\text{proper submodule}\},
    \end{equation*}
    ordered by inclusion. Then $0\in \mathcal{S}$. Every chain 
    \begin{equation*}
        N_1\subset N_2\subset \cdots\subset N_n\subset \cdots
    \end{equation*}
    in S eventually stabilizes since $M$ is finitely generated. Thus the union $N=\bigcup_{i}N_i$ is in $\mathcal{S}$.
    By Zorn's lemma, there is a maximal element $\mathcal{S}$. This completes the proof.
\end{proof}

\begin{lemma}
    If every submodule of $M$ is a direct summand, then any subquotient of $M$ also has this property.
\end{lemma}

\begin{proof}
    Suppose
    \begin{equation*}
        V\subset N\subset M,\quad M=N\oplus N'=V\oplus V'.
    \end{equation*}
    Then we have $N=V\oplus(V'\cap N)$, so submodules inherit this property.

    If a module has this property, then a quotient is isomorphic to a submodule.
    Hence, every subquotient of $M$ is isomorphic to a submodule. This completes the proof.
\end{proof}

\section{Semisimple Rings}

In this section, we give the classification of semisimple rings (definition \ref{def.semisimple}).

\begin{lemma}
    Let $R$ be a ring. Denote by $\RMod{R}$ the category of right $R$-modules. Let $\{e_{ij}\}$ be the standard basis of $\Mat{n}(R)$. 
    Then the functors
    \begin{equation*}
        M \longmapsto M^{\oplus n},\quad Ne_{11}\longmapsfrom N
    \end{equation*}
    gives an equivalence of categories $\RMod{R}\simeq\RMod{\Mat{n}(R)}$.
\end{lemma}

\begin{proof}
    One way of composing the functors clearly gives (up to natural isomorphism) the identity functor $M\longmapsto M$. The other order gives
    \begin{equation*}
        N\longmapsto (Ne_{11})^{\oplus n}\simeq 
        \bigoplus_{i=1}^n Ne_{ii}\simeq N.
    \end{equation*}
    The last isomorphism follows from the fact that $e_{ii}$ is a set of orthogonal idempotents in $\Mat{n}(R)$ that sums to $1$.
\end{proof}

\begin{corollary}\label{cor.matDss}
    Let $D$ be a division ring. 
    Then $\Mat{n}(D)$ is a semisimple ring, and every simple right $\Mat{n}(D)$-module is isomorphic to $D^n$.
\end{corollary}

\begin{proof}
    We know $\RMod{D}$ pretty well; it's called linear algebra.
\end{proof}

Recall the following result.
\begin{proposition}\label{prop.prodRings}
    Let $R=\prod_{i}R_i$ be a finite product of rings. Let $e_i$ be the corresponding idempotents and $\pi_i$ be the projections onto $R_i$. 
    Then the functors
    \begin{equation*}
        M \longmapsto (Me_i)_i,\quad 
        \bigoplus_{i}\pi_i^*N_i \longmapsfrom (N_i)_i
    \end{equation*}
    give an equivalence of categories $\RMod{R}\simeq\prod_{i}\RMod{R_i}$.
    Here $\pi_i^*$ denotes the pullback functor along $\pi_i$.
\end{proposition}

\begin{proof}
    One direction of composing the functors is trivial.
    The other is easy once idempotents are understood.
\end{proof}

\begin{lemma}
    If $f:R\to S$ is a surjective ring homomorphism, then the pullback functor $f^*:\RMod{S}\to\RMod{R}$ sends simple $S$-modules to simple $R$-modules.
\end{lemma}

\begin{proof}
    Any $R$-submodule has to be an $S$-submodule as well.
\end{proof}

\begin{theorem}\label{thm.ssring}
    Semisimple rings. TFAE:
    \begin{enumerate}[nosep]
        \item $R$ is a semisimple ring.
        \item $R$ is semisimple as a module over itself.
        \item $R$ has a factorization as in (\ref{eq.ssring}), i.e., a \emph{finite} product of matrix algebras over division rings.
    \end{enumerate}
\end{theorem}

\begin{proof}
    1 $\Rightarrow$ 2: trivial.

    2 $\Rightarrow$ 3: $R^\op=\End_R(R)$. If we know that as a module $R$ is only a \emph{finite} direct sum of simple modules, then (\ref{eq.ssring}) follows. But this is clear, as $R$ is finitely generated.

    3 $\Rightarrow$ 1: By Corollary \ref{cor.matDss}, $R$ is a finite product of semisimple rings.
    We wish to show that the second functor in Proposistion \ref{prop.prodRings} sends (a vector of) semisimple modules to semisimple $R$-modules.
    
    Each $\pi_i$ is surjective, so $\pi_i^*$ preserves simple modules by the above lemma.

    It's well-known that the pullback commutes with both limits and colimits because they have both left and right adjoints (the two 'base change' functors). In particular, $\pi_i^*$ commutes with direct sums.
    This completes the proof.
\end{proof}

\section{Wedderburn's Theorem}

\begin{definition}
    A ring $R$ is simple if it has not nontrivial two-sided ideals.
\end{definition}

\begin{remark}
    Obviously, a semisimple ring need not be simple, since we've shown that it can be a product of two rings.

    However, a simple ring $R$ need not be semisimple either.
    Two examples are provided below.
\end{remark}

\begin{definition}
    A ring $R$ is left (resp. right) Artinian if it is Artinian as a left (resp. right) $R$-module.
\end{definition}

\begin{theorem}[Wedderburn]
    TFAE:
    \begin{enumerate}[nosep]
        \item $R$ is simple and semisimple.
        \item $R$ is a matrix algebra over a division ring.
        \item $R$ is simple and right (or left) Artinian.
        \item $R$ is semisimple and all simple $R$-modules are isomorphic.
    \end{enumerate} 
\end{theorem}

\begin{proof}
    1 $\Rightarrow$ 2 is clear. 2 $\Rightarrow$ 4 is Corollary \ref{cor.matDss}.

    2 $\Rightarrow$ 3: $R=\Mat{n}(D)$ is a finite-dimensional $D$-vector space, hence Artinian.

    3 $\Rightarrow$ 1: Since $R$ is right (say) Artinian, it has a minimal right ideal $\mathfrak{a}$.
    Then $R\mathfrak{a}$ is a nonzero two-sided ideal, hence $R=R\mathfrak{a}$.
    Therefore, as an $R$-module, $R$ is a quotient of $\bigoplus_{r\in R}\mathfrak{a}$, hence semisimple by Corollary \ref{cor.sssubquot} and Therorem \ref{thm.ssring}.

    4 $\Rightarrow$ 2: The isotypic decomposition of $R_R$ only has one term, so $R^\op=\End_R(R)$ is a matrix algebra over a division ring.
\end{proof}

\begin{example}
    Any simple ring that's not Artinian is therefore not semisimple.
    
    Let $k$ be a field, $E$ be a $k$-vector space of countably infinite dimension, and $R=\End_k(E)$.
    Notice that $m=\{f\in R:\rk(f)<\infty\}$ is a maximal \emph{two-sided} ideal. Indeed, if $f\in R\setminus m$, then $f$ induces an isomorphism between two subspaces, both of which are isomorphic to $E$. Hence $1\in RfR$ and $m$ is maximal. We see that $R/m$ is a simple ring.

    However, $R/m$ is clearly not left Artinian. Let $\{e_1,e_2,\cdots\}$ be a basis of $E$. Then
    \begin{equation*}
        I_k=m+\left\{f\in R:k!\nmid n\Longrightarrow f(e_n)=0\right\}
    \end{equation*}
    is a descending chain of left ideals of $R$ which does not stabilize.
\end{example}

\begin{example}
    Another standard example is the Weyl algebra, defined by
    \begin{equation*}
        W = \C\langle x,\partial_x\rangle := T(V)/(p\otimes q-q\otimes p-1)
    \end{equation*}
    where $T$ denotes the tensor algebra, and $V$ is a $\C$-vector space with basis $\{p,q\}$. This can be viewed as the \emph{canonical commutation relation} from quantum mechanics. We see that
    \begin{itemize}[nosep]
        \item $W$ is spanned by $x^i{\partial_x}^j$ for $i,j\in\N$;
        \item $W$ is simple, seen by applying $[\partial_x,-]$ multiple times;
        \item $W$ is not a division ring, but it also has no nonzero zero-divisors. Easily seen by inspecting the leading term in $\partial_x$.
    \end{itemize}
    By clause 2 of Wedderburn's theorem, we conclude that $W$ cannot be semisimple.
\end{example}

\section*{The End}

This note partially follows \href{https://ocw.mit.edu/courses/18-706-noncommutative-algebra-spring-2023/mit18_706_s23_full_lec.pdf}{Noncommutative algebra}.

\noindent Compiled on \todayymd.

\noindent\home

\end{document}
